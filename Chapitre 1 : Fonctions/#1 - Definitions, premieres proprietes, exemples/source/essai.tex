\PassOptionsToPackage{unicode=true}{hyperref} % options for packages loaded elsewhere
\PassOptionsToPackage{hyphens}{url}
%
\documentclass[french,]{article}
\usepackage{lmodern}
\usepackage{amssymb,amsmath}
\usepackage{ifxetex,ifluatex}
\usepackage{fixltx2e} % provides \textsubscript
\ifnum 0\ifxetex 1\fi\ifluatex 1\fi=0 % if pdftex
  \usepackage[T1]{fontenc}
  \usepackage[utf8]{inputenc}
  \usepackage{textcomp} % provides euro and other symbols
\else % if luatex or xelatex
  \usepackage{unicode-math}
  \defaultfontfeatures{Ligatures=TeX,Scale=MatchLowercase}
\fi
% use upquote if available, for straight quotes in verbatim environments
\IfFileExists{upquote.sty}{\usepackage{upquote}}{}
% use microtype if available
\IfFileExists{microtype.sty}{%
\usepackage[]{microtype}
\UseMicrotypeSet[protrusion]{basicmath} % disable protrusion for tt fonts
}{}
\IfFileExists{parskip.sty}{%
\usepackage{parskip}
}{% else
\setlength{\parindent}{0pt}
\setlength{\parskip}{6pt plus 2pt minus 1pt}
}
\usepackage{hyperref}
\hypersetup{
            pdftitle={Les fonctions},
            pdfauthor={Fabien Delhomme},
            pdfborder={0 0 0},
            breaklinks=true}
\urlstyle{same}  % don't use monospace font for urls
\usepackage{longtable,booktabs}
% Fix footnotes in tables (requires footnote package)
\IfFileExists{footnote.sty}{\usepackage{footnote}\makesavenoteenv{longtable}}{}
\setlength{\emergencystretch}{3em}  % prevent overfull lines
\providecommand{\tightlist}{%
  \setlength{\itemsep}{0pt}\setlength{\parskip}{0pt}}
\setcounter{secnumdepth}{0}
% Redefines (sub)paragraphs to behave more like sections
\ifx\paragraph\undefined\else
\let\oldparagraph\paragraph
\renewcommand{\paragraph}[1]{\oldparagraph{#1}\mbox{}}
\fi
\ifx\subparagraph\undefined\else
\let\oldsubparagraph\subparagraph
\renewcommand{\subparagraph}[1]{\oldsubparagraph{#1}\mbox{}}
\fi

% set default figure placement to htbp
\makeatletter
\def\fps@figure{htbp}
\makeatother

\ifnum 0\ifxetex 1\fi\ifluatex 1\fi=0 % if pdftex
  \usepackage[shorthands=off,main=french]{babel}
\else
  % load polyglossia as late as possible as it *could* call bidi if RTL lang (e.g. Hebrew or Arabic)
  \usepackage{polyglossia}
  \setmainlanguage[]{french}
\fi

\title{Les fonctions}
\author{Fabien Delhomme}
\date{3 septembre 2018}

\begin{document}
\maketitle

\newpage

\textbf{\huge{Disclaimer}}

Bonjour à tous, ceci est le cours que je vais vous dispenser pendant
cette année de 2018--2019, avec pour objectif d'avoir le bac S.

Ce document est le support de cours. Il sera divisé en plusieurs
documents qui présenterons chacun à chapitres.

Il est \textbf{très important} de retenir que ce document est encore en
construction, c'est-à-dire qu'il subsiste des fautes de typo, ou
mathématiques. Si vous en voyez, \textbf{il faut le signaler} pour que
je puisse le corriger au plus vite.

Trêve de parole, commençons !

\newpage

\hypertarget{quest-ce-quune-fonction}{%
\section{Qu'est-ce qu'une fonction ?}\label{quest-ce-quune-fonction}}

\hypertarget{motivations}{%
\subsection{Motivations}\label{motivations}}

Les fonctions permettent de modéliser au mieux des phénomènes physiques.
C'est un formalisme très utilisé en physique, biologie, et en
mathématiques.

Les fonctions sont aussi présentes au bac: il y a forcément au moins un
exercice qui porte sur les variations d'une fonctions.

\hypertarget{duxe9finition}{%
\subsection{Définition}\label{duxe9finition}}

Une fonction associe quelque chose avec quelque chose. Ici, on
s'intéresse uniquement aux \emph{fonctions numériques}. C'est-à-dire à
des fonctions qui associent des nombres réels à des nombres réels. On
verra dans un second temps quelques fonctions définies sur les
\emph{complexes} (notion que l'on verra plus tard).

\hypertarget{repruxe9sentation-dune-fonctions-numuxe9riques}{%
\section{Représentation d'une fonctions
numériques}\label{repruxe9sentation-dune-fonctions-numuxe9riques}}

On peut donc représenter une fonctions dans un repère orthonormé. Cela
permet de visualiser les fonctions, et surtout :

\begin{itemize}
\tightlist
\item
  De retenir certaines de leur propriétés
\item
  D'avoir un sens intuitif de leur croissance
\end{itemize}

On reparlera de cela prochainement, mais retenez qu'il est
\emph{essentiel} d'avoir en tête les principales fonctions qui sont au
programme pour avoir une intuition correcte.

\hypertarget{comment-repruxe9senter-une-fonction}{%
\subsection{Comment représenter une fonction
?}\label{comment-repruxe9senter-une-fonction}}

Il suffit de représenter tous les couples de points de la forme
\[ (x, f(x))\] avec \(x \in \mathbb{R}\).

\hypertarget{les-fonctions-de-base}{%
\section{Les fonctions de base}\label{les-fonctions-de-base}}

\hypertarget{motivations-1}{%
\subsection{Motivations}\label{motivations-1}}

Pour mieux appréhender le concept de fonction, il faut avoir en tête ces
exemples.

\hypertarget{les-fonctions-linuxe9aires-et-affines}{%
\subsection{Les fonctions linéaires et
affines}\label{les-fonctions-linuxe9aires-et-affines}}

Ce sont les deux types de fonctions les plus simples qui existent ! Ces
fonctions sont sont très importante, comme on le verra par la suite.

Ce sont les fonctions du type : \[f(x) = a *x\] Où \(a\) est une
constante, comme par exemple \[f(x) = 3*x\] où ici \(a\) vaut \(3\).

Ce genre de fonctions permet de modéliser des phénomènes dits linéaires.
Un exemple très concret peut être donnée par les soldes, ou par la
modélisation du prix d'un abonnement :

Imaginons qu'une entreprise propose un abonnement de ce type :

\begin{itemize}
\tightlist
\item
  Tout nouvel abonnement coute 10 euros
\item
  Puis, le client paye 3 euros par mois
\end{itemize}

Comment modéliser le prix en fonction du mois que coute cette abonnement
? Et bien, la réponse est donnée par la fonction \(f(x) = 10*x + 3\).

\textbf{Remarques :} Attention à la notation, normalement (c'est-à-dire
techniquement parlant), \(f(x)\) représente un nombre. De façon abusive
je noterai parfois dans ce cours \(f(x)\) pour parler d'une fonction. Je
vous dirai les endroits où il est important de faire la distinction
(surtout lors de la rédaction des copies).

La représentation graphique de cette fonction (c'est-à-dire sa
\emph{courbe représentative}) est une \emph{droite}! (cf exercices).

\hypertarget{fonction-carruxe9}{%
\subsection{Fonction carré}\label{fonction-carruxe9}}

Un pas vers une fonction plus compliquée ! C'est la fonction suivante :
\[f : x \longrightarrow x *x = x^2\]

Par exemple:

\begin{itemize}
\tightlist
\item
  \(f(1) = 1\)
\item
  \(f(2) = 2*2 = 4\)
\end{itemize}

\hypertarget{interlude-fonction-ruxe9ciproque}{%
\subsection{Interlude : fonction
réciproque}\label{interlude-fonction-ruxe9ciproque}}

On parle de réciproque \(g\), d'une fonction \(f\), une fonction telle
que, si : \[ y = f(x) \]

Alors : \[ x = g(y) \].

Les prochaines sections montrent des exemples de fonctions réciproques.

\hypertarget{fonction-racine-carruxe9e}{%
\subsection{Fonction racine carrée}\label{fonction-racine-carruxe9e}}

C'est la fonction réciproque de la fonction racine carré. Attention,
ici, j'ai omis par soucis de rapidité une tonne de détails techniques,
mais en gros, et nous y reviendrons plus tard de toute façon, la
fonction racine marche comme il suit :

\begin{itemize}
\tightlist
\item
  \(3^2 = 9\), donc \(\sqrt{9} = 3\)
\item
  \(4^2 = 16\), donc \(\sqrt{16} = 4\)
\item
  \(\sqrt{x}\) pour \(x\) négatif n'a pas vraiment de sens (pour
  l'instant !)
\item
  \(\sqrt{2}\) est un nombre bien mystérieux.. (Que nous essaierons de
  calculer, et voir quelques propriétés).
\end{itemize}

\hypertarget{fonction-inverse}{%
\subsection{Fonction inverse}\label{fonction-inverse}}

Fonction qui associe un nombre à son inverse, tout simplement. Par
exemple, \(f(3) = \frac{1}{3}\). On peut remarquer que \(f\) est sa
propre réciproque.

\hypertarget{les-polynuxf4mes}{%
\subsection{Les polynômes}\label{les-polynuxf4mes}}

Les polynômes sont les premières constructions à partir des briques de
fonctions élémentaires. On parle de polynôme dès que l'on a une fonction
du type \[ f(x) = a_n * x^n + \dots + a_1 *x + a_0 \] Autrement dit, dès
que l'on a une somme de puissance de \(x\) multipliés par des nombres
\(a_n, \dots, a_0\), c'est que l'on a affaire à des polynômes.

Les polynômes sont \textbf{très} importants pour le mathématicien, car
en un certain sens, ils permettent de tester des théorèmes (et même d'en
prouver) facilement, puis d'étendre la véracité de ces théorèmes à
toutes les fonctions possibles !

Faisons le point dès maintenant sur quelque chose qu'il faut à tout prix
maitriser : la recherche de racines d'un polynôme.

\textbf{Vocabulaire:} une \emph{racine} d'un polynôme \(P\) est un
nombre \(x\) tel que \(P(x) = 0\).

Au lycée, vous ne verrez jamais de polynôme de degré supérieure à 3.
C'est-à-dire que vous serrez confronter, au pire, à des \(x^3\) dans vos
calculs (et heureusement !)

On peut remarquer au passage que les fonctions affines, les fonctions
linéaires, et les fonctions constantes, sont des polynômes !

\hypertarget{trouver-les-racines-dun-polynuxf4me-de-second-degruxe9}{%
\subsubsection{Trouver les racines d'un polynôme de second
degré}\label{trouver-les-racines-dun-polynuxf4me-de-second-degruxe9}}

Voici un paragraphe du type " formulaire ", que je déteste faire car il
ne donne pas d'explication sur ce qui se passe, mais uniquement des
formules à retenir. Je prendrai néanmoins soin d'écrire des exercices
qui vous feront prouver les formules ci-dessous.

On se donne un polynôme de second degré : \[ f(x) = ax^2 + bx + c\]

Où \(a,b\) et \(c\) sont des nombres quelconques.

\textbf{Premier réflexe pour trouver les racines:} (tous en chœurs !)
\emph{trouver le discriminant !} Pour un polynôme de second degré, on a
la formule \[ \boxed{ \Delta = b^2 - 4ac }\]

Et les racines sont données par les formules (magique !)
\[ x_1 = \frac{ - b + \sqrt{\Delta}}{2a}\]
\[ x_2 = \frac{ - b - \sqrt{\Delta}}{2a}\]

\textbf{Conclusion:}

\begin{itemize}
\tightlist
\item
  Si \(\Delta\) est négatif, il \emph{n'y a pas de racines réelles} (ce
  qui ont vu les complexes comprennent pourquoi j'insiste sur le mot «
  réel » )
\item
  Si \(\Delta\) est nul, alors ``vous vous êtes fait avoir'', car il n'y
  a qu'une seule racine, et c'était une simple identité remarquable
  permettait de conclure. Vous inquiétez pas, cela arrive aux meilleurs.
\item
  Si \(\Delta\) est positif, on obtient deux racines distinctes.
\end{itemize}

\textbf{Astuce de pro:}

\begin{itemize}
\tightlist
\item
  Parfois, lorsque \(a =1\), on a un critère qui permet de trouver les
  racines d'un polynôme de second degré (voir exercice). En effet, il se
  trouve que si on note \(x_1\) et \(x_2\) les deux racines d'un
  polynôme, et que \(a=1\), alors : \[ x_1 x_2 = c\] \[ x_1 + x_2 = b\]
  Donc, à votre avis, quelles sont les racines du polynôme
  \(x^2 + 3x +2\) ?
\end{itemize}

\hypertarget{le-sens-de-variation-dun-polynuxf4me-de-second-degruxe9}{%
\subsubsection{Le sens de variation d'un polynôme de second
degré}\label{le-sens-de-variation-dun-polynuxf4me-de-second-degruxe9}}

Dans ce paragraphe, j'empiète un peu sur les notions présentées au
suivant. Mais je pense qu'il est important de tout mettre au même
endroit. Si vous ne savez plus ce que représente un tableau de
variation, alors regardez
{[}\^{}croissance-décroissance{]}(\#Croissance, décroissance).

On peut donc en déduire le sens de variation facilement. Il y a deux
cas.

\begin{itemize}
\tightlist
\item
  Soit \(a\) est positif, et alors \(f\) décroit de \(-\infty\) à
  \(\frac{b}{2a}\) et croit de \(-\frac{b}{2a}\) à \(+ \infty\).
\item
  Soit \(a\) est négatif, et c'est l'inverse..
\end{itemize}

Ce qui donne pour \(a\) positif :

\begin{longtable}[]{@{}ccc@{}}
\toprule
x & \(]-\infty \quad \frac{b}{2a}]\) &
\([\frac{b}{2a} \quad \infty[\)\tabularnewline
\midrule
\endhead
P(x) & \(\searrow\) & \(\nearrow\)\tabularnewline
\bottomrule
\end{longtable}

\textbf{Astuce de mémorisation:} On retrouvera ces résultats rapidement
grâce à la dérivée, ou en ayant en tête la représentation graphique d'un
polynôme.

\hypertarget{tableau-de-signe-dun-polynuxf4me-de-second-degruxe9.}{%
\subsubsection{Tableau de signe d'un polynôme de second
degré.}\label{tableau-de-signe-dun-polynuxf4me-de-second-degruxe9.}}

Plus important, le tableau de signe d'un polynôme de second degré peut
\emph{toujours} être trouvé rapidement, avec un peu d'entrainement. Il
ne doit donc pas vous posez de grandes difficultés.

Là encore, il y a deux cas, selon le signe de \(a\).

Dans le cas où \(a\) est positif :

\begin{longtable}[]{@{}llll@{}}
\toprule
\begin{minipage}[b]{0.09\columnwidth}\raggedright
x\strut
\end{minipage} & \begin{minipage}[b]{0.22\columnwidth}\raggedright
\(]-\infty,x_2]\)\strut
\end{minipage} & \begin{minipage}[b]{0.18\columnwidth}\raggedright
\([x_2 ,x_1]\)\strut
\end{minipage} & \begin{minipage}[b]{0.23\columnwidth}\raggedright
\([x_1, +\infty[\)\strut
\end{minipage}\tabularnewline
\midrule
\endhead
\begin{minipage}[t]{0.09\columnwidth}\raggedright
P(x)\strut
\end{minipage} & \begin{minipage}[t]{0.22\columnwidth}\raggedright
\(+\)\strut
\end{minipage} & \begin{minipage}[t]{0.18\columnwidth}\raggedright
\(-\)\strut
\end{minipage} & \begin{minipage}[t]{0.23\columnwidth}\raggedright
\(+\)\strut
\end{minipage}\tabularnewline
\bottomrule
\end{longtable}

Et c'est l'inverse quand \(a\) est négatif.

\textbf{Astuce de mémorisation :} Là encore, n'essayez pas de retenir ce
tableau par coeur, mais à la place, ayez en tête la courbe
représentative d'une fonction polynômiale simple comme \(x^2 -1\), qui a
pour racine \(x_1 = -1\) et \(x_2 =1\)\footnote{c'est une identité
  remarquable !}, qui vous permet de retrouver tous ces résultats ci
dessus !

\hypertarget{propriuxe9tuxe9s-des-fonctions}{%
\section{Propriétés des
fonctions}\label{propriuxe9tuxe9s-des-fonctions}}

Maintenant que l'on a toutes ses fonctions en tête, regardons les
principales propriétés de ces fonctions

\hypertarget{croissance-duxe9croissance}{%
\subsection{Croissance, décroissance}\label{croissance-duxe9croissance}}

Une fonction est dite \emph{croissante} (resp. \emph{décroissante}) sur
un intervalle \(I\) si et seulement si :
\[ \forall x, y \in I \quad x \geq y \iff f(x) \geq f(y) \] resp:
\[ \forall x, y \in I \quad x \leq y \iff f(x) \geq f(y) \]

On parle de \emph{stricte} croissante (resp. \emph{stricte}
décroissance) lorsque les propriétés énoncées plus haut sont vraies en
remplaçant une inégalité large par une inégalité stricte..

Pour résumer :

\begin{itemize}
\tightlist
\item
  Une fonction croissante conserve l'ordre des antécédents. C'est-à-dire
  que l'odre des images et des antécédents et le même.
\item
  Au contraire, une fonction décroissante reverse cet ordre.
\end{itemize}

\textbf{Points de vocabulaire} :

\begin{itemize}
\tightlist
\item
  Le symbole \(\forall\) signifie ``pour tout''. Ce n'est pas dans le
  programme, mais c'est sacrément pratique ! (Ça veut dire qu'il faut
  pas l'utiliser dans les copies).
\item
  L'image d'un point \(x\) par la fonction \(f\) est simplement
  \(f(x)\).
\item
  L'antécédent d'un point \(y\) par la fonction \(f\) est \textbf{un}
  (il est possible qu'il y en ait plusieurs) nombre \(x\) tel que
  \(f(x) = y\)
\end{itemize}

Souvent (au bac, dans la vie d'un mathématicien, d'un physicien, d'un
ingénieur etc..) ou veut connaître le \textbf{sens de variation} d'une
fonction. Cela revient à construite un tableau qui résume les endroits
(ensemble de nombres) où la fonctions croit ou lorsque la fonction
décroit.

Voir les exercices pour des exemples.

\hypertarget{signe-dune-fonction}{%
\subsection{Signe d'une fonction}\label{signe-dune-fonction}}

Le signe d'une fonction désigne assez simplement les endroits (les
ensembles de nombres, réunions d'intervalles) où la fonction est
positive, et les endroits où la fonction est négative.

De même, on cherche souvent à établir un \textbf{tableau de signe} d'une
fonction pour savoir les endroits où elle est positive ou négative.
D'ailleurs, souvent, le tableau de signe est plus simple à établir que
le tableau de variation. (Cela explique la puissance de la dérivée, qui
comme nous allons bientôt le voir, permet de ramener l'étude d'une
tableau de variation à un tableau de signe !!).

Mais d'abord, il nous faut comprendre le concept de limite.

\hypertarget{domaine-de-duxe9finition}{%
\section{Domaine de définition}\label{domaine-de-duxe9finition}}

Une domaine de définition d'une fonction \(f\) est l'ensemble des
nombres sur lequel la fonction \(f\) est définie.

\textbf{Exemple:} la fonction inverse admet pour domaine de définition
\(] -\infty; 0[ \cup ]0; \infty[\), car il n'existe pas d'inverse pour 0
( la division par 0 n'a pas de sens !!).

\textbf{Notes:} il faut \emph{toujours} commencer par se demander, même
si cela n'est pas écrit explicitement dans l'énoncé, quel est le domaine
de définition d'une fonction que l'on a sous les yeux. Cela permet
d'éviter bien des problèmes !

\textbf{Remarques: (à relire plus tard)} Vous pouvez maintenant
comprendre ce que j'ai passé sous silence avec les fonctions
réciproques. En effet, pour calculer la réciproque d'une fonction, il
faut toujours considérer son domaine de définition, et surtout son
domaine image. La réciproque de \(f\) aura comme ensemble de départ le
domaine image de \(f\). C'est pourquoi par exemple la fonction racine
carrée n'est pas défini pour des nombres réels négatifs (et pour les
petits malins du fond qui connaissent les complexes, notions au
programme de terminal S, sachez qu'on y reviendra !)

\end{document}
